\documentclass[a4paper,titlepage,12pt]{scrartcl}

% utf-8
\usepackage{polyglossia}
\setdefaultlanguage[babelshorthands]{ngerman}
\usepackage{fontspec}

% german names
\usepackage{ngerman}

% colored links
\usepackage{color}
\usepackage[colorlinks]{hyperref}
% custom colors
\definecolor{grey}{rgb}{0.2,0.2,0.2}
\definecolor{orange}{rgb}{1,0.3,0}
\definecolor{turqoise}{rgb}{0,0.7,0.5}

% code listings
\usepackage{listings}
\lstset{%
	basicstyle={\ttfamily \small},
	breaklines=true,
	commentstyle=\color{grey},
	keywordstyle=\color{orange},
	language=Go,
	numbers=left,
	showspaces=false,
	stringstyle=\color{turqoise},
	xleftmargin=20pt
}

% graphics
\usepackage{graphicx}
\graphicspath{i/}

% fancy headers and footers
\usepackage{fancyhdr}
\pagestyle{fancy}
% clear style
\fancyhead{}
\fancyfoot{}
% new style
\renewcommand{\headrulewidth}{0.5pt}
\renewcommand{\footrulewidth}{0.5pt}
\fancyhead[LE,RO]{\rightmark}
\fancyhead[LO,RE]{\leftmark}
\fancyfoot[LE,RO]{\thepage}
\fancyfoot[LO,RE]{Qualtitätssicherung WS 2012/2013}
\fancypagestyle{plain}{%
	\fancyhf{}
	\renewcommand{\headrulewidth}{0pt}
	\renewcommand{\footrulewidth}{0.5pt}
	\fancyfoot[LE,RO]{\thepage}
	\fancyfoot[LO,RE]{Qualitätssicherung WS 2012/2013}
}

% no indented paragraphs
\usepackage{parskip}

% TODO: what's this?
\setkomafont{disposition}{\normalfont\bfseries}

\usepackage{natbib}

\begin{document}

\titlehead{%
	\includegraphics[width=0.9\linewidth]{i/hska_logo.png}
}

\title{Ausarbeitung Qualitätssicherung}
\author{Florian Limberger \\ Mat.-Nr.: 30470}
\date{Wintersemester 2012/2013}
\publishers{%
    \textbf{Betreuer:} Prof.\,Dr. Dirk W. Hoffmann
}
\maketitle

\section{Einleitung}
\label{sec:intro}
Im Rahmen der Blockveranstaltung \emph{Qualtitätssicherung} war die Aufgabenstellung,
ein Werkzeug zum Verschmelzen zweier Textdateien auf der Basis einer gemeinsamen Version zu erstellen.
Dabei sollte ein Algorithmus zur Bestimmung der Längsten gemeinsamen Teilfolge verwendet werden,
welcher nicht rekursiv arbeitet und eine maximale Komplexität von $O(n^2)$ hat.
Dieses Werkzeug war im Anschluss unter Verwendung von Zeilenüberdeckung zu testen.
Die Implementierungssprache war freigestellt,
sofern Werkzeuge zur Zeilenüberdeckungsanalyse zur Verfügung stehen.

Diese Ausarbeitung bezieht sich auf das Programm \texttt{qsmerge},
welches die Lösung des Autors für die obige Aufgabenstellung darstellt.
Es wurde in C implementiert,
weshalb das Werkzeug \texttt{gcov} zu, Test der Zeilenüberdeckung verwendet wurde.

\section{Design}
\label{sec:design}
Die Benutzerschnittstelle des \texttt{qsmerge}-Programms wurde \texttt{diff3} nachempfunden,
welches üblicherweise in unixoiden Systemumgebungen zur Verfügung steht.
Daher handelt es sich um ein Kommandozeilenprogramm,
welches mit folgender Signatur aufgerufen wird:
\lstset{language=sh, numbers=none, xleftmargin=0pt}
\begin{lstlisting}
qsmerge dave.txt orig.txt mike.txt
\end{lstlisting}
Dabei handelt es sich bei \texttt{dave.txt} und \texttt{mike.txt} um die beiden neuen Dateiversionen handelt,
welche miteinander verschmolzen werden sollen,
während \texttt{orig.txt} die gemeinsame Ausgangsversion ist.
Bei einer vollständig automatisch durchgeführten Verschmelzung ist der Rückgabewert 0,
falls Konflikte auftraten 2.
In diesem Falle werden die Konflikte ebenfalls in die Ausgabe geschrieben,
wobei sich die Syntax an der Ausgabe von \texttt{diff3 -m} orientiert:
\begin{verbatim}
<<<<<<< dave.txt:1
/* Dave */
=======
/* Mike */
>>>>>>> mike.txt:1
\end{verbatim}
Im Gegensatz zum Vorbild werden keine Bereiche unterstützt,
da \texttt{qsmerge} lediglich auf Zeilenebene arbeitet.
Auch wird bei einem Konflikt nicht der Inhalt der Ursprungsdatei angezeigt,
da die Ausgabe dadurch sehr unübersichtlich werden würde.
Weitere Unterschiede zwischen \texttt{qsmerge} und \texttt{diff3} bestehen darin,
dass \texttt{qsmerge} keinerlei Optionen unterstützt und keine \texttt{ed}-Skripte ausgeben kann.
\\
Optionen zum automatischen Lösen von Konflikten durch Bevorzugung einer Version wären jedoch einfach hinzuzufügen.

\section{Algorithmus}
\label{sec:algorithm}
Verwendet wurde der in \citet{web:eppstein} beschriebene LCS-Algorithmus,
zusammen mit einem eigenen Merge-Algorithmus.

\section{Implementiation}
\label{sec:implementation}
\texttt{Qsmerge} wurde in C implementiert.

\section{Test}
\label{sec:test}

\bibliographystyle{plainnat}
\bibliography{algorithms}

\end{document}
