\documentclass[a4paper,titlepage,12pt]{scrartcl}

% utf-8
\usepackage{polyglossia}
\setdefaultlanguage[babelshorthands]{ngerman}
\usepackage{fontspec}

% german names
\usepackage{ngerman}

% colored links
\usepackage{color}
\usepackage[colorlinks]{hyperref}
% custom colors
\definecolor{grey}{rgb}{0.2,0.2,0.2}
\definecolor{orange}{rgb}{1,0.3,0}
\definecolor{turqoise}{rgb}{0,0.7,0.5}

% code listings
\usepackage{listings}
\lstset{%
	basicstyle={\ttfamily \small},
	breaklines=true,
	commentstyle=\color{grey},
	keywordstyle=\color{orange},
	language=C,
	numbers=left,
	showspaces=false,
	stringstyle=\color{turqoise},
	xleftmargin=20pt
}

% graphics
\usepackage{graphicx}
\graphicspath{i/}

% fancy headers and footers
\usepackage{fancyhdr}
\pagestyle{fancy}
% clear style
\fancyhead{}
\fancyfoot{}
% new style
\renewcommand{\headrulewidth}{0.5pt}
\renewcommand{\footrulewidth}{0.5pt}
\fancyhead[LE,RO]{\rightmark}
\fancyhead[LO,RE]{\leftmark}
\fancyfoot[LE,RO]{\thepage}
\fancyfoot[LO,RE]{Qualtitätssicherung WS 2012/2013}
\fancypagestyle{plain}{%
	\fancyhf{}
	\renewcommand{\headrulewidth}{0pt}
	\renewcommand{\footrulewidth}{0.5pt}
	\fancyfoot[LE,RO]{\thepage}
	\fancyfoot[LO,RE]{Qualitätssicherung WS 2012/2013}
}

% no indented paragraphs
\usepackage{parskip}

% TODO: what's this?
\setkomafont{disposition}{\normalfont\bfseries}

\usepackage{natbib}

\usepackage{amsfonts}

\usepackage{verbatim}

\begin{document}

\titlehead{%
	\includegraphics[width=0.9\linewidth]{i/hska_logo.png}
}

\title{Ausarbeitung Qualitätssicherung}
\author{Florian Limberger \\ Mat.-Nr.: 30470}
\date{Wintersemester 2012/2013}
\publishers{%
    \textbf{Betreuer:} Prof.\,Dr. Dirk W. Hoffmann
}
\maketitle

\tableofcontents

\clearpage

\section{Einleitung}
\label{sec:intro}
Im Rahmen der Blockveranstaltung \emph{Qualtitätssicherung} war die Aufgabenstellung,
ein Werkzeug zum Verschmelzen zweier Textdateien auf der Basis einer gemeinsamen Version zu erstellen.
Dabei sollte ein Algorithmus zur Bestimmung der Längsten gemeinsamen Teilfolge verwendet werden,
welcher nicht rekursiv arbeitet und eine maximale Komplexität von $O(n^2)$ hat.
Dieses Werkzeug war im Anschluss unter Verwendung von Zeilenüberdeckung zu testen.
Die Implementierungssprache war freigestellt,
sofern Werkzeuge zur Zeilenüberdeckungsanalyse zur Verfügung stehen.

Diese Ausarbeitung bezieht sich auf das Programm \texttt{qsmerge},
welches die Lösung des Autors für die obige Aufgabenstellung darstellt.
Es wurde in C implementiert,
weshalb das Werkzeug \texttt{gcov} zu, Test der Zeilenüberdeckung verwendet wurde.
\\
Der gesamte Quellcode für das Programm sowie diese Ausarbeitung is im Internet unter der Adresse \url{https://github.com/flimberger/qsmerge} als Git-Repository zu finden.

\section{Design}
\label{sec:design}
Die Benutzerschnittstelle des \texttt{qsmerge}-Programms wurde \texttt{diff3} nachempfunden,
welches üblicherweise in unixoiden Systemumgebungen zur Verfügung steht.
Daher handelt es sich um ein Kommandozeilenprogramm,
welches mit folgender Signatur aufgerufen wird:
\lstset{language=sh, numbers=none, xleftmargin=0pt}
\begin{lstlisting}
qsmerge dave.txt orig.txt mike.txt
\end{lstlisting}
Dabei handelt es sich bei \texttt{dave.txt} und \texttt{mike.txt} um die beiden neuen Dateiversionen handelt,
welche miteinander verschmolzen werden sollen,
während \texttt{orig.txt} die gemeinsame Ausgangsversion ist.
Bei einer vollständig automatisch durchgeführten Verschmelzung ist der Rückgabewert 0,
falls Konflikte auftraten 1.
In diesem Falle werden die Konflikte ebenfalls in die Ausgabe geschrieben,
wobei sich die Syntax an der Ausgabe von \texttt{diff3 -m} orientiert:
\begin{verbatim}
<<<<<<< dave.txt:1
/* Dave */
=======
/* Mike */
>>>>>>> mike.txt:1
\end{verbatim}
Die Ausgabe wird in den Standardausgang geschrieben,
wodurch man sie einfach in Dateien oder andere Programme umleiten kann.
Falls während dem Programmablauf Fehler auftreten,
wird es mit einem Status von 2 beendet.
Durch diese Architektur kann \texttt{qsmerge} einfach in Skripten verwendet werden.

Im Gegensatz zum Vorbild werden keine Bereiche unterstützt,
da \texttt{qsmerge} lediglich auf Zeilenebene arbeitet.
Auch wird bei einem Konflikt nicht der Inhalt der Ursprungsdatei angezeigt,
da die Ausgabe dadurch sehr unübersichtlich werden würde.
Dafür wird die Zeile zusätzlich zur Ursprungsdatei angegeben.
Weitere Unterschiede zwischen \texttt{qsmerge} und \texttt{diff3} bestehen darin,
dass \texttt{qsmerge} keinerlei Optionen unterstützt und keine \texttt{ed}-Skripte ausgeben kann.
\\
Optionen zum automatischen Lösen von Konflikten durch Bevorzugung einer Version wären jedoch einfach hinzuzufügen.

\section{Algorithmus}
\label{sec:algorithm}
Der Verschmelzungsalgorithmus basiert auf der längsten gemeinsamen Teilfolge von Zeilen aller drei Dateiversionen und der darauffolgenden Auflösung der Konflikte.
Für die weitere Beschreibung gelten folgende Konventionen:
\begin{itemize}
\item $\mathbb{L}$ sei die Menge aller geordneten Listen von Textzeilen.
\item Großgeschriebene Variablen stehen für Elemente von $\mathbb{L}$,
wobei $A, B, O \in \mathbb{L}$ die beiden Eingabedateien und die Originaldatei repräsentieren.
\item $|L| \in \mathbb{N}_0$ mit $L \in \mathbb{L}$ beschreibt die Kardinalität von $L$,
also die Anzahl der Textzeilen in $L$.
\item $L(i)$ mit $i \in \mathbb{N}_0, L \in \mathbb{L}$ beschreibt die $i$-te Zeile aus $L$.
\item $\mbox{LCS}: \mathbb{L} \times \mathbb{L} \rightarrow \mathbb{L}$ beschreibt eine Funktion,
welche die längte gemeinsame Teilfolge der Zeilen der beiden Eingabeparameter zurückgibt.
\end{itemize}

Um die LGT aller drei Dateien zu erhalten,
werden zunächst die längsten gemeinsamen Teilfolgen mit $L_{A,O} = \mbox{LCS}(A, O)$ und $L_{B,O} = \mbox{LCS}(B, O)$ berechnet,
mit $L_{A,B,O} = \mbox{LCS}(L_{A,O}, L_{B,O})$ wird anschließend die LGT aller drei Eingabedateien berechnet.

Zur Ermittelung der längsten gemeinsamen Teilfolge wurde der in \citet{web:eppstein} beschriebene LGT-Algorithmus verwendet,
jedoch musste er an die Eigenheiten des Programms angepasst werden.
Der Algorithmus nutzt aufsteigende dynamische Programmierung,
um Rekursion zu vermeiden.
Dabei werden alle Zeilen der Eingabeparameter miteinander verglichen,
indem über eine Matrix iteriert wird,
in welcher die Vergleichsresultate der einzelnen Zeilen gespeichert sind.
Die Iteration beginnt rechts unten in der ausgenullten Matrix und wird nach links oben fortgesetzt.
Mit jeder Übereinstimmung zweier Zeilen erhält das aktuelle Feld den Wert des diagonal darunterliegenden Feldes um 1 erhöht.
Alle nachfolgenden Felder erhalten den Wert der Felder rechts daneben oder darunter,
außer es kommt zu einer weiteren Übereinstimmung.
Dadurch erhält man die Anzahl der gleichen Zeilen im ersten Feld der Mastrix.
Daraufhin kann von vorne durch die Matrix iteriert werden,
wobei die diagonalen Übergänge zwischen den Feldwerten gesucht werden,
da an diesen Stellen die Zeilen übereinstimmen.

Mit den auf diese Weise ermittelten LGT wird die Ausgabe zusammengestellt,
wofür über alle Zeilen der beiden neuen Dateiversionen iteriert wird.
Solange sowohl $A(i)$ als auch $B(j)$ gültige Zeilen beschreiben,
also wenn $i < |A| \wedge j < |B|$ gilt,
wird die Ausgabe von folgende Regeln bestimmt:
\begin{itemize}
\item Wenn $A(i) = B(j)$ gilt,
dann wird eine der beiden Zeilen (welche identisch sind) ausgegeben.
\item Wenn $A(i) = L_{A,B,O}(m)$ gilt,
dann wird die Zeile $B(j)$ ausgegeben,
da diese in B neu eingefügt wurde.
\item Wenn $B(j) = L_{A,B,O}(m)$ gilt,
dann wird $A(i)$ augegeben.
\item Wenn $A(i) = L_{A,O}(k)$ gilt,
dann wird die Zeile $B(j)$ ausgegeben,
da diese in A beibehalten wurde.
Die Zeile $A(i)$ wird verworfen.
\item Wenn $B(j) = L_{B,O}$ gilt,
dann wird $A(i)$ ausgegeben und $B(j)$ verworfen.
\item Im letzten Fall liegt ein Verschmelzungskonflikt vor,
da sowohl $A(i)$ als auch $B(j)$ neu eingefügt worden.
Daher werden beide Zeilen mit der in Abschnitt~\ref{sec:design} beschriebenen Notation ausgegeben.
\end{itemize}
Wenn obige Bedingung nicht erfüllt ist,
dann ist die eine Datei länger als die andere,
also werden alle folgenden Zeilen ausgegeben.

\section{Implementation}
\label{sec:implementation}
Wie bereits in Abschnitt~\ref{sec:intro} erwähnt,
wurde \texttt{Qsmerge} in C implementiert.
Um die Verarbeitung zu beschleunigen,
werden die Zeilen vor dem Verarbeiten mit dem SHA1-Hashalgorithmus gehashed.
Dies erleichtert auch die Speicherverwaltung,
da die \emph{message digest} alle 20 Bytes groß sind.
So hängt die Größe des benötigten Speichers nur von der Anzahl der Zeilen ab,
und nicht zusätzlich von der Länge jeder Zeile.
\\
Der in Abschnitt~\ref{sec:algorithm} beschriebene Algorithmus musste abgewandelt werden,
da die Anzahl der Textzeilen bzw. der Hashes in \texttt{size\_t}-Variablen gespeichert sind,
und daher keine negativen Werte annehmen können.
Daher müssen die Indices für das Array des LGT-Algorithmus umgerechnet werden,
da in \citet{web:eppstein} der Zähler für die Iteration bis unter 0 laufen.
\\
Eine weitere Änderung ist die Verwendung eindimensionaler anstatt mehrdimensionaler Arrays.
Dies vereinfacht die Speicherverwaltung ebenfalls,
da nur ein Zeiger in den Speicher verwaltet werden muss.
Ein weiterer Vorteil ist die bessere Performance gegenüber mehrdimensionaler Arrays,
da diese zu \emph{cache thrashing} neigen.
Ein Nachteil dieses Schemas ist jedoch ein komplizierterer Zugriff,
da die Zeilen- und Spaltenindices auf einen einzigen Index umgerechnet werden müssen.
Dies ist jedoch ein wohlbekanntes Problem,
so dass der Nachteil der schlechteren Lesbarkeit als weniger schwerwiegend betrachtet wird.

Der Quellcode wurde auf drei Dateien aufgeteilt,
um die unmittelbar für \texttt{qsmerge} benötigten Anteile von allgemeineren zu trennen:
\begin{itemize}
\item Fehlerbehandlungsfunktionen befinden sich in \texttt{error.c},
ihre Signaturen sind in \texttt{error.h} definiert.
\item Speicherverwaltungsfunktionen befinden sich in \texttt{fmalloc.h},
ihre Signaturen sind in \texttt{fmalloc.h} definiert.
\item der übrige Code befindet sich in \texttt{qsmerge.c}.
Da sich die \texttt{main}-Funktion hier befindet,
verfügt sie nicht über eine separate Header-Datei.
\end{itemize}

Die Datei \texttt{error.h} stellt eine Mini-Bibliothek zur Fehlerbehandlung dar,
da sie diverse Fehlerbehandlungsfunktionen zur Verfügung stellt.
Von diesen wird jedoch lediglich die Funktion \texttt{die()} verwendet,
welche eine Fehlermeldung ausgibt und das Programm mit dem Status 2 beendet.
Wenn der Fehlerstring mit einem Doppelpunkt endet,
dann wird eine Stringrepräsentation des aktuellen Inhalts der globalen \texttt{errno}-Variablen nach dem Fehlerstring ausgegeben.
Anschließend wird immer ein \emph{newline}-Zeichen ausgegeben,
so dass der Fehlerstring nicht auf ein solches enden muss.
\\
Die große Menge der nicht verwendeten Funktionen rührt daher,
dass die gleiche oder zumindest sehr ähnliche Versionen von \texttt{error.c} in diversen anderen Projekten des Autors eingesetzt werden.

In der Datei werden Funktionen zur Speicherverwaltung deklariert.
Im Grunde handelt es sich lediglich um Umhüllungen für die Funktionen der Standardbibliothek,
die das Programm bei nicht erfolgreicher Speicherreservierung mit einer Fehlermeldung beenden.
Für diesen Zweck wird die \texttt{die}-Funktion aus \texttt{error.c} verwendet.
\\
Die zur Verfügung stehenden Funktionen sind:
\begin{itemize}
\item \texttt{fcalloc()}, welches die Funktion \texttt{calloc} aus der Standardbibliothek kapselt
\item \texttt{fmalloc()}, welches \texttt{malloc} umhüllt
\item \texttt{frealloc()}, um \texttt{realloc} zu ersetzen
\end{itemize}
Das Präfix emph{f} steht dabei für \emph{failsafe},
da diese Funktionen aus Sicht des Programms nicht fehlschlagen können.

Der übrige Quellcode befindet sich in \texttt{qsmerge.c},
da er größtenteils direkt mit dem Programm zusammenhängt.
\\
Zunächst werden die zwei Konstanten \texttt{BUFSIZE} und \texttt{SHA1dlen} in einem \texttt{enum} deklariert,
damit sie im Debugger mit ihren Symbolischen Namen zur Verfügung stehen
(im Unterschied zu mittels \texttt{\#define} deklarierten Konstanten).
\texttt{BUFSIZE} definiert Größe aller im Programm verwenden Puffer,
\texttt{SHA1dlen} die Größe eines \emph{message digest} des SHA1-Hashalgorithmus.
\\
Das folgende Makro \texttt{HASHSIZE} vereinfacht das Berechnen der Größe eines Arrays von SHA1-\emph{message digest}-Elementen.
\\
Danach werden zwei Datenstrukturen definiert:
\texttt{Hashtab} kapselt die Liste von Hashes.
Tatsächlich handelt es sich um vergrößerbare Arrays,
oft auch Vektoren genannt.
Die Struktur hat drei Elemente: \texttt{\_maxcnt}, \texttt{curcnt} und \texttt{data}.
In \texttt{\_maxcnt} ist die Gesamtgröße des Datenspeichers hinterlegt.
Diese Variable ist ``privat'',
was durch den vorangestellten Unterstrich signalisiert wird.
Sie ist lediglich für interne Verwaltungsaufgaben,
genauer gesagt die Vergrößerung des Arrays,
gedacht.
\\
In \texttt{curcnt} ist die Anzahl der belegten Elemente gespeichert.
\\
\texttt{Data} ist schließlich der Zeiger auf das Array selbst,
welches separat alloziert wird.
\\
Die zweite Datenstruktur ist \texttt{File} und dient der Zuordnung von Dateinamen und \texttt{FILE}-Zeigern.
\lstset{language=C,numbers=left,xleftmargin=20pt}
\begin{lstlisting}
typedef struct {
	size_t _maxcnt;	/* maximum count of hashes */
	size_t curcnt;	/* current count of hashes */
	uchar *data;	/* hash data */
} Hashtab;

typedef struct {
	FILE *fp;
	char *name;
} File;
\end{lstlisting}

Nach den Definitionen folgen die Funktionsdeklarationen,
welche alle statisch sind,
da sie dadurch in ihrem Gültigkeitsbereich auf die Quelldatei beschränkt werden.
Dadurch kommt es zu Compilerwarnungen,
wenn eine Funktion deklariert ist,
nicht aufgerufen wird.
Dadurch wird die Wartung erleichtert,
da nicht verwendeter Code einfacher entfernt werden kann.

Die beiden Funktionen \texttt{fileopen} und \texttt{fileclose} dienen der Verwaltung von \texttt{File}-Strukturen.
\\
Die nächste Funktion, \texttt{hash}, akzeptiert Zeiger auf eine \texttt{Hashtab}- und eine \texttt{File}-Struktur als Argumente.
Es wird erwartet,
dass die \texttt{Hashtab}-Struktur uninitialisiert ist,
dies wird kedoch nicht sichergestellt.
Da die Verwendung jedoch rein intern ist,
sollte dies kein großes Problem darstellen.
Die Funktion reserviert Speicher für die Hashes,
und liest die Datei,
welche durch die \texttt{File}-Struktur spezifiziert ist,
zeilenweise ein.
Jede Zeile wird mit dem SHA1-Algorithmus gehashed,
wozu die libtomcrypt-Bibliothek benutzt wird (\citet{lib:tomcrypt}.
Sollte die Anzahl der Elemente im Array nicht ausreichen,
wird es mittels \texttt{frealloc} vergrößert.
Sollten während der Eingabe Fehler aufgetreten sein,
wird das Programm mit einem Aufruf von \texttt{die} beendet.

Darauf folgen drei weitere kurze Helferfunktionen,
welche die Handhabung der SHA1-\emph{message digests} vereinfachen.
\texttt{Gethash} erwartet einen Zeiger auf eine Hashliste sowie einen Index als Parameter,
und gibt einen Zeiger auf das durch den Index spezifizierte Element des Arrays zurück.
Während viele der anderen Funktionen ihre Eingabeparameter nicht auf ihre Gültigkeit prüfen,
wird dies von \texttt{gethash} getan.
Der Grund dafür ist,
dass die Hashlisten teilweise in deutlich größeren Speichersegmenten liegen,
als sie eigentlich benötigen und daher Zeiger,
welche hinter das letzte Listenelement zeigen,
dennoch auf gültige Speicheradressen verweisen.
Implementierungsfehler in den verwendenten Algorithmen wären so schwerer zu entdecken,
da sie zwar falsche Ergebnisse liefern,
aber das Programm ansonsten weiterläuft.
Daher überprüft \texttt{gethash},
ob der Index auf ein gültiges Listenelement verweist,
andernfalls wird das Programm beendet.
\\
Die Funktion \texttt{hashequals} vergleicht die beiden Hashes,
auf welche die Argumente zeigen.
\\
Mit \texttt{copyhash} wird der Hash,
auf welchen der zweite Parameter \texttt{source} zeigt,
an die durch den ersten Parameter \texttt{target} angegeben ist.
Die Zeiger auf Hashes,
welche die letzten beiden Funktionen als Argumente verwenden,
sollten durch Aufrufe der ersten Funktion, \texttt{gethash},
erlangt werden.
Dies gilt vor allem für den \texttt{target}-Parameter der Funktion \texttt{copyhash},
da es sonst zu Speicherzugriffsverletzungen kommen kann.

In der Funktion \texttt{findcls} wird der in Abschnitt~\ref{sec:algorithm} beschriebene LGT-Algorithmus implementiert.
Die Argumente sind drei Zeiger auf \texttt{Hashtab}-Strukturen,
wobei das erste Argument auf eine bislang uninitialisierte Struktur zeigt und in welcher die Ergebnisliste gespeichert wird.
Die anderen beiden Argumente zeigen auf bereits initialisierte Listen,
welche die Eingangsparameter sind.
\\
Wie am Anfang des Kapitels bereits erwähnt wurde,
musste er an die Implementierung angepasst werden,
da die Länge der Hashliste in einem nicht vorzeichenbehafteten Zahlentypen gespeichert ist.
Dies führt dazu,
dass die Funktion der Zählervariablen von der Funktion der Indexvariablen getrennt wird und es daher vier Variablen anstatt zweien gibt.
Während in \citet{web:eppstein} von $|L|$ abwärts bis einschließlich Null gezählt wird,
zählt diese Implementierung von Null bis $L$.
Damit die Reihenfolge,
in welcher das Array durchlaufen wird die gleiche ist,
wie im ursprünglichen Algorithmus,
werden die Indices \texttt{i2} und \texttt{j2} durch die Subtraktion der Zählervariablen \texttt{i} und \texttt{j} von der jeweiligen Anzahl der Listenelemente errechnet.
\\
Nachdem die Matrix für den LGT-Algorithmus aufgebaut wurde,
wird der Speicherplatz für die Ausgangsliste reserviert.
Als Größe wird die Längere der kürzeren Eingabeliste verwendet,
da die längste gemeinsame Teilfolge maximal so lang sein kann,
wie die kürzere der beiden Eingabefolgen.
Anschließend werden die Hashes,
welche zur LGT der beiden Eingabelisten gehören,
in die Ausgabeliste kopiert.

Die Funktion \texttt{merge} übernimmt die eigentliche Funktion des Programmes.
Ihre Parameter sind die \texttt{File}-Strukturen,
welche auf die Originaldatei und die beiden zu verschmelzenden Dateien verweisen.
Der Rückgabewert ist die Anzahl der Verschmelzungskonflikte.
\\
Das Namensschema der lokalen Variablen ist sehr kompakt,
aber dennoch logisch:
\texttt{o}, \texttt{a} und \texttt{b} sind die Hashlisten der drei Eingabedateien und entsprechen $O$, $A$ und $B$.
\texttt{A1} und \texttt{b1} sind die Hashlisten der LGT der jeweiligen Eingabedatei mit der Originaldatei,
entsprechen also $L_{A,O}$ und $L_{B,O}$.
In \texttt{lcs} ist schließlich die LGT aller drei Dateien gespeichert,
wodurch sie $L_{A,B,O}$ entspricht.
Da über alle Elemente jeder dieser Listen iteriert werden muss,
wird für jede Liste ein Zähler benötigt.
Daher ist jede Zählervariable mit dem Namen der zugehörigen Liste mit einem angehängten \texttt{cnt} benannt,
mit Ausnahme von \texttt{lcnt},
welches zu \texttt{lcs} gehört.
In der Variable \texttt{errs} wird die Anzahl der Verschmelzungskonflikte gespeichert,
ihr Wert wird am Ende der Funktion zurückgegeben.
\\
Im Ablauf der Funktion werden zunächst alle Zeilen der Eingabedateien mittels \texttt{hash} gehashed und in den oben beschriebenen Variablen gespeichert,
danach werden durch \texttt{findlcs} alle LGT berechnet.
Danach werden die Dateizeiger der zu verschmelzenden Dateien mittels \texttt{rewind} auf den Anfang gesetzt,
da die für die Ausgabe die Zeilen in Klartext benötigt werden.
Die folgende Schleife läuft solange noch nicht alle Zeilen der Eingabedateien eingelesen wurden.
In ihr wird die Ausgabe nach den in Abschnitt~\ref{sec:algorithm} beschriebenen Regeln bestimmt.
\\
Zur Zeileneingabe dient die Variable \texttt{buf},
welche auch wieder mittels \texttt{printf} ausgegeben wird.
Eine Alternative wäre die Verwendung von \texttt{puts},
jedoch befindet sich ein \emph{newline}-Zeichen am Ende der Eingabezeile,
so dass sich bei der Verwendung von \texttt{puts} zwei davon am Ende jeder Ausgabezeile befinden würden,
weshalb das letzte Zeichen der Eingabezeile entfernt werden müsste.
Um die günstigere der beiden Varianten herauszufinden,
müsste die Laufzeit jeder der beiden Varianten gemessen werden.
Im Rahmen der Veranstaltung war die \texttt{printf}-Version einfacher und klarer,
da sie weniger Verwaltungsarbeit und damit Quellcode benötigt.
\\
Nachdem alle Zeilen der zu verschmelzenden Dateien verarbeitet sind,
wird der Speicher aller Hashlisten freigegeben und die Funktion kehrt mit der Rückgabe der Anzahl aller Verschmelzungskonflikte zurück.

In der Hauptroutine des Programms werden lediglich die Argumente geparsed,
die Eingabedateien geöffnet und die Funktion \texttt{merge} aufgerufen.
Aufgrund deren Rückgabewert wird der Rückgabestatus des Programms bestimmt.
Vor dem Beenden des Programms werden noch die zu Beginn geöffneten Dateien geschlossen.

\section{Test}
\label{sec:test}
Die eigentliche Vorgabe für die Tests war das Erreichen von 100\% Zeilenabdeckung. 
Einige Fälle sind jedoch äußerst schwierig zu testen,
da sie sehr schwer herbeizuführen sind.
Dies bezieht sich vor allem auf die Standardein- und -ausgabefunktionen.
So ist beispielsweise das Herbeiführen des Falls,
dass das Schließen einer erfolgreich geöffneten Datei fehlschlägt,
sehr schwierig.
Gleiches gilt für Lesefehler auf dem \texttt{FILE}-Zeiger.
\\
Eine weitere Schwierigkeit war die \texttt{gethash}-Funktion.
Sie überprüft,
ob ihr \texttt{size}-Argument innerhalb der Liste liegt,
aus welcher sie ein Element zurückgeben soll,
was beim Schreiben des Verschmelzungsalgorithmus hilfreich war.
Bei korrekter Funktionsweise des Algorithmus wird jedoch nie ein zu großer Index übergeben,
so dass die Abbruchbedingung im Zeilenüberdeckungstest nie ausgeführt wird.
Dies hätte man durch das Entfernen dieser Überprüfung lösen können,
jedoch könnte es dann zu Schwierigkeiten kommen,
wenn der Algorithmus verändert werden soll.
Daher wird in diesem Fall auf eine Abdeckung verzichtet,
da ansonsten sinnvolle Funktionalität entfernt werden müsste.
\\
Um die Reaktion auf Speicherfehler zu testen,
wurde der ausführenden Shell mit der \texttt{ulimit}-Funktion der zur Verfügung stehende Speicher begrenzt.
Hierfür musste experimentell die Grenze bestimmt werden,
bei welcher die verlinkten Programmbibliotheken sowie das Programm selbst in den Speicher geladen werden können,
dieser jedoch nicht mehr für die dynamische Speicherreservierung ausreicht.
Auf dem Laptop des Autors mit einem 64bit Linux betrug dieses Limit 4400 kB.
\\
Damit wurde die Abbruchbedingung der \texttt{fmalloc}-Funktion einmal erfüllt,
da die Fehlerbehandlung der übrigen Speicherverwaltungsfunktionen identisch ist wurde auch an dieser Stelle auf eine 100\%-Abdeckung verzichtet.

In allen der oben genannten Fällen blieben Ausführungen der \texttt{die}-Funktion ungetest,
welche jedoch bereits an anderen Stellen getestet wurde.
Aus diesem Grund wurde das Fehlen von 100\% Zeilenabdeckung als vertretbar eingestuft.

\clearpage
\appendix

\section{Manual}
\label{apx:man}
\verbatiminput{qsmerge.man}

\bibliographystyle{plainnat}
\bibliography{algorithms,libraries}

\end{document}
